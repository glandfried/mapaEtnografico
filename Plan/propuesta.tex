\documentclass{article}
\usepackage[utf8]{inputenc}
\input{../aux/encabezado.tex}

\title{Mapa de sitios de campo etnográfico: \\ \large popuesta para el Colegio de Graduados de Antropología}
\author{Laura Raquel Amaya Botello$^{1,3}$ y Gustavo Landfried$^{2,3}$}
\affil{\small 1 Universidad Nacional de Colombia. 2 Universidad de Buenos Aires. 3 Creencias Adaptativas}
\affil[]{Correspondencia: \url{lramayab@unal.edu.co}}

\date{Abril 2021}

\begin{document}

\maketitle

\begin{abstract}
  La antropología social tiene en el mundo varias redes de colaboración para el trabajo de campo etnográfico comparativo.
  En la Argentina el desarrollo de este tipo de redes se ve limitada en parte por la falta de información al respecto, actualmente no contamos con los datos de los grupos de investigación que se encuentran realizando trabajo de campo en sitios indígeno-campesino en el país.
  Desde \emph{Creencias Adaptativas} nos proponemos desarrollar el mapa de sitios de campo etnográfico indidígeno/campesino, una base de datos pública que sirva de base para la colaboración entre los mismos profesionales activos del área.
  A su vez, esta información servirá como material para promover la actividad de la antropología en los medios de comunicación masivos.
\end{abstract}

\section*{Propuesta de trabajo y colaboraciones}

Desde \emph{Creencias Adaptativas} nos proponemos impulsar el desarrollo de un mapa de los grupos de investigación que se encuentran realizando trabajo de campo en sitios indígeno-campesino en el país.
El objetivo es poveer una base de datos pública que de base para la colaboración entre los mismos profesionales activos del área y como material para la difusión de nuestra profesión en los medios de comunicación.

Nuestra metodología está orientada a la aplicación de protocolos colaborativos de trabajo basados en los principios de desarrollo de código abierto (open source): accesible, interoperable, reproducible y reutilizable.
Bajo estos criterios de trabajo, ofrecemos al Colegio de Graduados en Antropología la colaboración de nuestro equipo, \emph{Creencias Adaptativas}, durante el desarrollo del mapa.

En la primera fase, el proyecto tiene por objetivo registrar únicamente los sitios de campo etnográfico indígeno-campesino de la Argetnina con al menos 3 a 5 años de continuidad, excluyendo por el momento los sitios de campo arqueológico, urbanos y los úbicados en países limítrofes.
La plan de recolección de datos consiste en realizar entrevistas cortas con al menos una persona representante de cada uno de los grupos de investigación.
Antes de comenzar se elaborará una encuesta base que contenga las preguntas mínimas que deben realizarse a todos los grupos.
La estrategia para registrar a todos los grupos activos que quieran participar de esta iniciativa se conoce como metodología de "bola de nieve": se le pide a cada grupo entrevistado referencias de otros grupos.
Esta estrategia es compatible con la realización de una llamado público.

El proyecto estará accesible en un repoisitorio público de control de versiones (github), que se utiliza habitualmente para el desarrollo colaborativo de software.











\end{document}
